\documentclass{jsarticle}
\usepackage{listings}
\usepackage{color}
\lstset{%
 language={C},
 backgroundcolor={\color[gray]{.95}},%
 basicstyle={\small},%
 identifierstyle={\small},%
 commentstyle={\small\itshape},%
 keywordstyle={\small\bfseries},%
 ndkeywordstyle={\small},%
 stringstyle={\small\ttfamily},
 frame={shadowbox},
 breaklines=true,
 columns=[l]{fullflexible},%
 numbers=left,%
 xrightmargin=0zw,%
 xleftmargin=3zw,%
 numberstyle={\scriptsize},%
 stepnumber=1,
 numbersep=1zw,%
 lineskip=-0.5ex%
}

\begin{document}

\section{全体のテンプレ}
別にこれに固定しなくてもよい
\begin{lstlisting}
/*
input data
*/

void init(){
}

bool input(){
}


/*
dp とか
*/
void init_solve(){
}

void q_input(){
}

int solve(){
	return 0;
}


int main(){
	int m;
	for(int i=0;i<m;i++){
		cout<<solve()<<endl;
	}
}
\end{lstlisting}
\section{include}
\begin{lstlisting}
#include<cmath>
#include<cassert>
#include<string>
#include<vector>
#include<queue>
#include<stack>
#include<set>
#include<map>
#include<algorithm>
#include<utility>
using namespace std;

\end{lstlisting}
\clearpage
\section{素数}
\begin{lstlisting}
vector<int> prime;
vector<int> prime_list;
void prime_set(int n){
	n+=100;
	prime.resize(n);
	for(int i=0;i<n;i++){
		prime[i]=1;
	}
	prime[0]=prime[1]=0;
	for(int i=0;i*i<n;i++){
		if(prime[i]){
			for(int j=i*2;j<n;j+=i){
				prime[j]=0;
			}
		}
	}
	for(int i=0;i<n;i++){
		if(prime[i])prime_list.push_back(i);
	}
}
\end{lstlisting}
\section{文字列}
\begin{lstlisting}
string revStr(string s){
    return string(s.rbegin(),s.rend());
}
\end{lstlisting}
\clearpage
\section{二次元幾何}
\begin{lstlisting}
#define X first
#define Y second

typedef long double ld;
typedef long long ll;
typedef ll P_type;
typedef pair<ll,ll> P;

const ld INF = 1e39;
const ld EPS = 1e-8;
const ld  PI = acos(-1);

P_type out_pro(P a,P b) {
  return (a.X * b.Y - b.X * a.Y);
}

P_type in_pro(P a,P b){
  return (a.X * b.X + a.Y *b.Y);
}

//以下2つは小数点以下が必要なことがあるのでldにする
ld pow_len(P a) {
  return a.X * a.X + a.Y * a.Y;
}

ld len(P a){
  return sqrt(pow_len(a));
}

P make_v(P a,P b) {
  return P(a.X - b.X, a.Y - b.Y);
}
//凸法
vector<P> convex(vector<P> list) {
  int m=0;
  vector<P> res(list.size()*2);
  sort(list.begin(),list.end());
  for(int i=0; i<list.size(); res[m++]=list[i++]){
    for(;m>1&&out_pro(make_v(res[m-1],res[m-2]),make_v(list[i],res[m-2]))<=EPS;--m);
  }
  for(int i=list.size()-2,r=m; i>=0; res[m++]=list[i--]){
    for(;m>r&&out_pro(make_v(res[m-1],res[m-2]),make_v(list[i],res[m-2]))<=EPS;--m);
  }
  res.resize(m-1);
  return res;
}
\end{lstlisting}

\end{document}

